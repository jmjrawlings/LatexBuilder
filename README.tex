\documentclass{article}
\usepackage{booktabs}
\usepackage{amsmath}

\title{LatexBuilder}
\author{Justin Rawlings}
\date{Sunday, 8 September 2024}
\begin{document}
\maketitle

\section{Introduction}
This is an introduction.
\section{Sections}
Document levels are maintained internally by the \emph{LatexDocument}.
So there is no need to remember which section/subsection/paragraph level
you are at.

Document scopes are implemented using readonly ref structs so they should
quite performant to use which can be a concern when writing large documents.

\subsection{A subsection}
This is now a subsection as we were already inside a Section.
\subsection{Another subsection}
\subsubsection{A subsubsection}
\paragraph{A paragraph}
\subparagraph{A subparagraph}
\subparagraph{Nesting maxes out at subparagraph}
That wasnt so bad.
\section{Some math}
The string interpolation features in newer versions of C\# is
very well suited to mixing code variables and latex markup.

We seek to calculate the value for $\gamma$:
\begin{align*}
    \gamma & = \alpha + \beta \tag{a tag} \\
    \alpha & = 1                          \\
    \beta  & = 2                          \\
    \gamma & = 1 + 2                      \\
           & = 3
\end{align*}
\section{Other things}
\subsection{Lists}
\begin{itemize}
    \item First item\begin{itemize}
              \item Nested item 1\item Nested item 2
          \end{itemize}
    \item Second item
\end{itemize}
\subsection{Tables}
\begin{table}[h]
    \centering
    \begin{tabular}{r|r|r}
        \toprule
        n & variable & value \\
        \midrule
        1 & $\alpha$ & $1$   \\
        2 & $\beta$  & $2$   \\
        3 & $\gamma$ & $3$   \\
        \bottomrule
    \end{tabular}
    \caption*{A table caption}
    \label{table}
\end{table}
\subsection{Manual section}
You can also manage sections explicitly using \emph{BeginSection} and \emph{EndSection}

\end{document}